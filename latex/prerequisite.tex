\chapter*{Prerequisite}

The material for this course assumes some basic assumptions about the course participants.
The participants in the audience should have some familiarity with concepts such as \textit{Metaverse},  Telelife~\citep{10.3389/frvir.2021.763340}, virtual reality and augmented reality.
Participants can be from various technical backgrounds but are willing to advance their understanding of the optimization side of displays, cameras or perception.
The course material will describe the base hardware used in the demonstrated optimizations.
However, the lecturers will not cover how to build this hardware from the ground up. 
Instead, they will provide a quick overview and cite relevant resources from the most recent literature.
The attendees are also expected to be knowledgeable or willing to learn Python programming language, modern machine learning libraries, signal theory, and optics.
But most importantly, above-average interest in optimizing future's devices is a must.

The code material of this course is a result of various research works from the \href{https://complightlab.com}{Computational Light Laboratory} that was conducted in the passing one-year timeframe (2021-2022). 
These research works include new methods for foveated rendering and image statistics~\citep{walton2021beyond}, learned techniques for holographic light transport~\citep{kavakli2022learned}, learned optimizations for multiplane holography~\citep{kavakli2022realisticdefocus}, perceptually guided hologram generation routines in displays~\citep{walton2021metameric} and learned optimizations for lensless cameras~\citep{kingshott2022unrolled}.

Individuals willing to replicate the outcome of our materials from this course on their local machines have to install several pieces of software in their operating systems.
Such individuals must be familiar with the \texttt{Python} programming language, and they should install \href{https://www.python.org/}{Python} and \href{https://jupyter.org/}{Jupyter Notebooks} in their operating systems.
In addition, these individuals will need to install \href{https://pytorch.org/}{Torch} with its Python bindings, \href{https://matplotlib.org/}{Matplotlib} and \href{https://plotly.com/}{plotly} libraries for plotting purposes while using the provided Jupyter Notebooks. 
When we compiled this material, our production machines used the Python distribution \texttt{3.9.7}, Torch distribution \texttt{1.9.0}, Matplotlib distribution \texttt{3.3.4} and Jupyter Notebook distribution \texttt{6.2.0}. As a final piece, please make sure to install our
library using:

\begin{tcolorbox}[breakable, size=fbox, boxrule=1pt, pad at break*=1mm,colback=cellbackground, colframe=cellborder]
\prompt{In}{incolor}{1}{\boxspacing}
\begin{Verbatim}[commandchars=\\\{\}]
\PY{n}{pip} \PY{n}{install} \PY{n}{odak}
\end{Verbatim}
\end{tcolorbox}

At the time of this course, Odak is at version \texttt{0.2.0}.
The participants willing to go beyond this course and learn more about relevant research are welcome to reach out to the lecturers via email.
In addition, Computational Light Laboratory offers \href{https://complightlab.com/seminars/}{seminars} from experts on relevant topics.
Finally, Computational Light Laboratory also invites all attendees to a research hub formulated as a \href{https://complightlab.com/research_hub/}{Slack Group}.
This way, curious readers and attendees of our course can keep up-to-date and meet more folks in the relevant fields.
