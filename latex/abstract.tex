\chapter*{Preface}
The evolution of the internet is underway, where immersive virtual 3D environments (commonly known as $\textit{metaverse}$ or $\textit{telelife}$) will replace flat 2D interfaces.
Crucial ingredients in this transformation are next-generation displays and cameras representing genuinely 3D visuals while meeting the human visual system's perceptual requirements.

This course will provide a fast-paced introduction to optimization methods for next-generation interfaces geared towards immersive virtual 3D environments.
Firstly, we will introduce lensless cameras for high dimensional compressive sensing (e.g., single exposure capture to a video or one-shot 3D).
Our audience will learn to process images from a lensless camera at the end.
Secondly, we introduce holographic displays as a potential candidate for next-generation displays.
By the end of this course, you will learn to create your 3D images that can be viewed using a standard holographic display.
Lastly, we will introduce perceptual guidance that could be an integral part of the optimization routines of displays and cameras.
Our audience will gather experience in integrating perception to display and camera optimizations.

This course targets a wide range of audiences, from domain experts to newcomers.
To do so, examples from this course will be based on our in-house toolkit to be replicable for future use.
The course material will provide example codes and a broad survey with crucial information on cameras, displays and perception.

